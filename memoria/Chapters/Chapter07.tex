\chapter{Contribución individual}
A lo largo del desarrollo del trabajo la carga de tareas a sido equitativa, trabajando de forma paralela cada uno de los integrantes en distintas tareas. A continuación se detalla la aportación individual de cada uno de los autores de este trabajo.
\section{Adrián}
\subsubsection{Investigación}
\begin{itemize}
	\item 
\end{itemize}
\subsubsection{Análisis de requisitos}
\begin{itemize}
	\item 
\end{itemize}
\subsubsection{Implementación}
\begin{itemize}
	\item 
\end{itemize}
\subsubsection{Pruebas}
\begin{itemize}
	\item 
\end{itemize}
\subsubsection{Memoria}
\begin{itemize}
	\item 
\end{itemize}

\section{Jesús Martín González}
\subsubsection{Investigación}
En la fase de investigación todos los integrantes del grupo comenzamos a introducirnos en los aspéctos básicos que rodean ya no sólo al \textit{browser fingerpriting} sino en general a las formas de detección de huellas digitales. En mi caso, ya contaba con una serie de nociones básicas sobre estos procedimientos, ya que soy un entusiasta de privacidad en dispositivos digitales. Aun así, mis conocimientos no iban mucho más allá de haber explorado distintas extensiones de navegador que podían evadir ciertas técnicas de detección de huella digital. \par

La fase de investigación fue muy general, ya que iniciamos una búsqueda de todo tipo de material entre todos que nos pudiera ser de utilidad, como herramientas, bibliotecas y demás. En mi caso me centré en recopilar \textit{papers} y documentación de interés, que luego han sido aprovechados para ayudarnos a escribir nuestra memoria. Esto me ayudó a comprender la diferencia entre las técnicas activas y pasivas de rastreo de huella digital y a distinguir las ventajas e inconvenientes que plantea frente al uso de cookies. Por último, también colaboré en la elaboración de una lista de posibles tecnologías a usar en nuestro proyecto. \par

\subsubsection{Análisis de requisitos}

En la fase de análisis de requisitos nos pusimos a valorar las distintas opciones que barajábamos, y decidir así con cuales nos quedaríamos. Mi tarea consistió en realizar pruebas de concepto con diversas tecnologías de servidores web, concretamente con Django y Node.js. Finalmente descartamos ambas y nos quedamos con Apache, como hemos explicado en la sección~\ref{subsec:rejected}. Además, también planteé la adopción del sistema de base de datos MongoDB. La conclusión fue que una base de datos relacional se adaptaba mejor a nuestras necesidades, por lo que también descartamos esta opción. Elegimos PHP y JavaScript, como los lenguajes de programación que utilizaríamos. Aunque ya había trabajado con ellos anteriormente, hacia bastante tiempo que no los utilizaba así que en mi caso también tuve que ponerme al día y repasar conceptos que tenía olvidados. \par 

También se analizó qué componentes del navegador recolectar con JavaScript a través de WebAPI. Me ocupé de identificar cómo podríamos obtener información relacionada con el protocolo de red, como es la IP origen, los DNS utilizados o información sobre la WebAPI \texttt{WebRTC}. Descartamos recopilar estos atributos al no considerarlos tan relevantes como otras características del propio navegador. Por último, me encargué de analizar cuales son los elementos obtenidos desde distintas fuentes que se repetían. La intención era comprobar si arrojaban valores diferentes para detectar posibles técnicas de evasión. Esta funcionalidad se ha dejado como trabajo futuro. \par 

\subsubsection{Implementación}

Mi primera labor relacionada con la implementación del proyecto fue entender la estructura de la aplicación, que había sido diseñada por Rubén. Este primer esquema se basaba en la primera prueba de concepto que había sido realizada por Rubén, Adrián y Samuel utilizando Apache y PHP. Como ya he comentado anteriormente, me llevó un tiempo el volver a familiarizarme con el lenguaje PHP. Mi primera contribución de código fue colaborar con Rubén para hacer la extracción de datos de la cabecera HTTP lo más flexible posible. Una vez me iba poniendo más al día con el lenguaje, pude contribuir a más tareas. Tras esto, colaboré en la extración de atributos con JavaScript a través de WebAPIs: primero mejorando la detección de fuentes del navegador, tarea que había sido iniciada por Adrián; y posteriormente mediante la detección de elementos Flash. Esta tarea fue desarrollada junto con Rubén, ya que él mejoró bastante mi primera implementación inicial. \par 

Mi contribución más importante en este apartado es el cálculo del porcentaje de similaridad que hay entre el usuario que accede a la página web y el resto de valores de los usuarios que han accedido anteriormente. Lo interesante de la implementación es que se pueden reutilizar secciones del código y consultas SQL que en un principio estaban destinadas sólo para identificar si un usuario había visitado nuestra plataforma y para pintar los valores de los atributos. Más tarde, mejoré esta funcionalidad protegiendo las consultas ante posibles ataques de \textit{SQLInjection}. También he contribuido a mejorar otras funcionalidades de la aplicación en momentos puntuales. Otras tareas también fueron llevadas a cabo pero finalmente no se adoptaron, como la generación de certificados SSL en local. \par 

\subsubsection{Pruebas}

Hacer pruebas con sistemas no tan comunes para el público general fue una de mis tareas a realizar. En concreto, se efectuaron pruebas desde los sistemas operativos Ubuntu y Debian, y desde los navegadores Brave, Waterfox, Tor Browser y Lynx, además de pruebas puntuales desde la herramienta cURL. El objetivo era verificar si nuestra aplicación era capaz de funcionar fuera de la plataforma XAMPP en Windows y obtener información de los navegadores menos comunes. También colaboré en alojar nuestra aplicación en una web de hosting, aunque sin éxito. Rubén retomó esta tarea y la llevó a cabo de forma satisfactoria. En la fase de pruebas tan solo encontré pequeños fallos no muy importantes que fueron subsanados.

\subsubsection{Memoria}

Respecto a la memoria, me he encargado de escribir el capítulo~\ref{ch:preliminares} de Preliminares. A la hora de repartir el trabajo de de la memoria, pensamos que iba a ser más interesante que me encargara yo de este capítulo porque algunos de los documentos académicos que habiamos recopilando anteriormente ya me resultaban familiares y me iba a ser más sencillo redactar los antecedentes del \textit{browser fingerprinting}. También he participado en la elaboración del capítulo~\ref{ch:conclusiones} de Conclusiones junto con Samuel. Tal como nos habíamos asignado los capítulos, tenía sentido que nosotros dos comenzaramos a redactarlo, enunciando los resultados finales y que lo concluyeran Adrián y Rubén con más detalles sobre el trabajo futuro de la implementación y sobre los problemas con los que nos habíamos encontrado. \par 

Por último, además de participar en la revisión del resto de partes redactadas por mis compañeros, destacar también mi colaboración en la traducción al inglés de los capítulos~\ref{ch:introduccion} y ~\ref{ch:conclusiones} de Introducción y Conclusiones, respectivamente. Aunque esta tarea ha sido realizada por Samuel, yo he prestado apoyo introduciendo algunas mejoras en la traducción. \par

\section{Rubén Peña García}
\subsubsection{Investigación}
En esta fase del proyecto comencé informándome acerca de los distintos métodos de \textit{browser fingerprinting} que existen, ya que mi conocimiento de la materia era casi nulo. Principalmente examiné el funcionamiento de otras aplicaciones capaces de obtener la huella digital del navegador, como \textit{AmIUnique}\cite{amiunique} o \textit{Panopticlick}\cite{panop_paper}. También me documenté acerca de los usos del reconocimiento de la huella digital de navegadores y su impacto gracias al estudio publicado por la Agencia Española de Protección de Datos \cite{aepd}.\par
Después mi investigación consistió en comprobar y listar todos los elementos que íbamos a obtener en nuestra aplicación y su método de obtención, para poder determinar el alcance de nuestro proyecto.\par
Para los datos que obtenemos de las cabeceras HTTP analicé los elementos que conformaban las cabeceras HTTP en los navegadores más utilizados, como \textit{Google Chrome}, \textit{Microsoft Edge}, \textit{Microsoft Internet Explorer} o \textit{Mozilla Firefox}. De todos los resultados obtenidos listé los más comunes para su almacenamiento en nuestra base de datos.\par
También investigué elementos que podíamos obtener mediante el uso de JavaScript, comprobando en otras aplicaciones los atributos que obtenían mediante esta tecnología. Algunas, como \textit{AmIUnique}, listan todos los elementos para que el usuario pueda verlos, pero en muchos casos no explican su método de obtención. Mi investigación en este ámbito consistió en examinar los distintos atributos que obtienen estas aplicaciones por un lado y encontrar su forma de obtención por otra parte. Además investigué distintos objetos de JavaScript, examinando los métodos de sus clases por si alguno podía servirnos a la hora de identificar a un usuario añadiendo información a su huella.
\subsubsection{Análisis de requisitos}
Una vez tuvimos claro el funcionamiento que tendría nuestra aplicación y la manera de obtener los datos para generar la huella investigamos que tecnologías utilizaríamos. Repartimos la investigación de tecnologías, sobre todo en lado de servidor, ya que estaba claro que en cliente tendríamos que utilizar HTML y JavaScript.\par
Mi tarea fue investigar \textit{frameworks} JavaScript para el servidor. El que más estudié fue \textit{Node.JS}\cite{nodejs}, ya que es uno de los \textit{frameworks} más extendidos y famosos. Al concluir todas nuestras investigaciones y poner nuestras conclusiones en común decidimos utilizar \textit{Apache} para nuestro servidor, ya que ofrecía muchas comodidades y todos los integrantes del trabajo habíamos trabajado con el anteriormente, lo que nos permitía trabajar de manera inmediata sin necesidad de aprender a utilizar un \textit{framework} desconocido.
\subsubsection{Implementación}
Esta etapa del proyecto fue la más larga y compleja. Al comienzo de esta me encargué de diseñar la estructura de una aplicación de prueba para obtener datos del navegador. Se trataba de una versión básica de lo que finalmente terminaría siendo nuestra aplicación. En ella programé lo necesario para identificar cada uno de los atributos de las cabeceras HTTP y mostrarlos por pantalla. Todo este código se almacenó en un repositorio de \textit{GitHub} que yo creé.\par
Posteriormente diseñé e implementé la base de datos con PhpMyAdmin, que en principio solo almacenaba en una única tabla el ID y los elementos de la cabecera HTTP. Esta base de datos iría aumentando su número de tablas y atributos a lo largo del desarrollo, ya que añadíamos las columnas según íbamos implementando los métodos de obtención de atributos de la aplicación.\par
Tras terminar la versión de prueba comenzamos la programación de los métodos encargados de obtener información mediante JavaScript. Me encargué de seguir desarrollando la estructura del código de la aplicación, de forma que este fuese lo más modular posible para que cada uno de los integrantes pudiésemos trabajar en paralelo de la forma más cómoda posible.\par
Después de completar la estructura del código comenzamos a programar las funciones para la obtención de elementos con JavaScript. Los integrantes del grupo nos fuimos repartiendo estas tareas, las cuales teníamos listadas desde la etapa de investigación. Cada uno programábamos por separado estas funciones y la íbamos añadiendo al código de la aplicación según las íbamos terminando. Entre otros me encargué de muchos de los elementos obtenidos del objeto \textit{Navigator}, del objeto \textit{Screen}, de \textit{Canvas}, de algunos plugins como \textit{AdBlock} o \textit{Flash}, la batería, el objeto \textit{Window}, los codecs de audio y vídeo o los dispositivos del sistema.\par
Una vez consideramos que no era necesario la obtención de más atributos para nuestra aplicación pasamos a implementar el resto de funcionalidades de nuestra aplicación. Entre estas funcionalidades yo me dediqué a programar el código necesario para la obtención la unicidad de la huella del usuario. También añadí la herramienta de información que hay en nuestra aplicación, la cual detalla la utilidad de cada uno de los elementos analizados en la página. Además me encargué de que el código en cliente pudiese enviar la información al servidor de manera asíncrona para no tener que recargar la página.\par 
Aunque el encargado de la apariencia estética de la aplicación fue Samuel, yo también programé parte de esta, cambiando algunos colores y aplicando la misma apariencia al apartado de gráficas de nuestra aplicación. 
\subsubsection{Pruebas}
Aunque durante el desarrollo del código de la aplicación fuimos comprobando que el código funcionase de manera correcta, fue cuando terminamos de programar cuando nos encontramos con varios problemas. Durante estas pruebas encontramos el que posiblemente era el mayor problema, la asincronía. Debido a la forma en la que realizábamos la asincronía, en algunas ocasiones se ejecutaban las llamadas asíncronas fuera del orden deseado, lo cual afectaba al resultado de la unicidad del usuario. Yo me ocupé de solucionar este problema, haciendo que las llamadas asíncronas se ejecutasen en orden.
\subsubsection{Memoria}
Para la realización de la memoria me dediqué a buscar una plantilla de \LaTeX que nos sirviese de base para la escritura de esta memoria. Me encargué modificar algunas cosas y adaptarla a nuestro gusto. Una vez lista nos repartimos la escritura de los capítulos entre los integrantes del grupo. Yo me encargué de la escritura del capítulo 5 y parte del capítulo 8.\par
Además de la escritura de estos capítulos, todos los integrantes hemos ido revisando los capítulos realizados por el resto de compañeros. De esta forma hemos podido añadir información que se nos hubiese podido olvidar individualmente y corregir errores en común.

\section{Samuel}
\subsubsection{Investigación}
\begin{itemize}
	\item 
\end{itemize}
\subsubsection{Análisis de requisitos}
\begin{itemize}
	\item 
\end{itemize}
\subsubsection{Implementación}
\begin{itemize}
	\item 
\end{itemize}
\subsubsection{Pruebas}
\begin{itemize}
	\item 
\end{itemize}
\subsubsection{Memoria}
\begin{itemize}
	\item 
\end{itemize}

\noindent
